\documentclass[12pt]{article}
\usepackage[left=1.2in, right=0.6in, top=0.9in, bottom=0.6in]{geometry}
\usepackage{amsmath}
\usepackage{booktabs}
\usepackage{titlesec}

% Reduce space above section titles
\titlespacing*{\section}{0pt}{4pt}{4pt}
\titlespacing*{\subsection}{0pt}{4pt}{2pt}

\setlength{\parskip}{4pt}
\setlength{\parindent}{0pt}

\title{\vspace{-2em}Statistics and Probability: Expectation vs Mean\vspace{-1em}}
\date{}

\begin{document}
\maketitle

\section*{1. Expectation vs Mean}

\begin{tabular}{@{}lll@{}}
\toprule
\textbf{Aspect} & \textbf{Expectation (Expected Value)} & \textbf{Mean (Average)} \\
\midrule
Context & Probability theory & Statistics and data analysis \\
Definition & Theoretical average over many trials & Empirical average of sample \\
Symbol & \( E[X], \mu \) & \( \bar{x} \) \\
Based on & Probability distribution & Observed data \\
Use case & Predict long-run average & Summarize actual dataset \\
\bottomrule
\end{tabular}

\section*{2. When Expectations Are Used}

\begin{itemize}
  \item Probability theory and stochastic modeling
  \item Decision-making under uncertainty (e.g., expected profit/loss)
  \item Games of chance and gambling
  \item Used in machine learning, Bayesian inference, economics
  \item Basis for variance, covariance, and moments
\end{itemize}

\section*{3. Mathematical Definitions}

\textbf{Expected Value (Discrete)}:
\[
E[X] = \sum_{i} x_i \cdot P(x_i)
\]

\textbf{Sample Mean}:
\[
\bar{x} = \frac{1}{n} \sum_{i=1}^{n} x_i
\]

\section*{4. Examples}

\subsection*{Example 1: Die Roll (Expectation)}

Rolling a fair six-sided die:

\[
E[X] = \frac{1 + 2 + 3 + 4 + 5 + 6}{6} = 3.5
\]

\textbf{Interpretation:} The expected value of a fair die roll is 3.5 (theoretical average over time).

\subsection*{Example 2: Observed Data (Mean)}

Sample: goals in 5 football matches: \([1, 2, 0, 3, 2]\)

\[
\bar{x} = \frac{1 + 2 + 0 + 3 + 2}{5} = \frac{8}{5} = 1.6
\]

\textbf{Interpretation:} This is the actual average of the observed sample.

\subsection*{Example 3: Lottery Ticket}

Ticket costs \$2. Possible outcomes:

\begin{tabular}{lll}
\toprule
Outcome & Prize (\$) & Probability \\
\midrule
Win big & 500 & 0.001 \\
Win small & 20 & 0.01 \\
Break even & 2 & 0.05 \\
Lose & 0 & 0.939 \\
\bottomrule
\end{tabular}

\[
\begin{aligned}
E[X] &= (500 - 2)(0.001) + (20 - 2)(0.01) + (2 - 2)(0.05) + (0 - 2)(0.939) \\
&= 0.498 + 0.18 + 0 - 1.878 = -1.2
\end{aligned}
\]

\textbf{Expected loss: \$1.20 per ticket}

\subsection*{Example 4: Business Product Launch}

\begin{tabular}{lll}
\toprule
Scenario & Profit (\$) & Probability \\
\midrule
High demand & 100{,}000 & 0.3 \\
Medium demand & 40{,}000 & 0.5 \\
Low demand & -20{,}000 & 0.2 \\
\bottomrule
\end{tabular}

\[
\begin{aligned}
E[\text{Profit}] &= (100{,}000)(0.3) + (40{,}000)(0.5) + (-20{,}000)(0.2) \\
&= 30{,}000 + 20{,}000 - 4{,}000 = 46{,}000
\end{aligned}
\]

\textbf{Expected profit: \$46,000}

\subsection*{Example 5: Insurance Premium}

Premium = \$100. Possible outcomes:

\begin{tabular}{lll}
\toprule
Incident & Payout (\$) & Probability \\
\midrule
No incident & 0 & 0.95 \\
Screen damage & 150 & 0.03 \\
Lost phone & 600 & 0.02 \\
\bottomrule
\end{tabular}

\[
E[\text{Payout}] = (0)(0.95) + (150)(0.03) + (600)(0.02) = 0 + 4.5 + 12 = 16.5
\]

\[
\text{Expected profit} = 100 - 16.5 = 83.5
\]

\textbf{Expected profit per policy: \$83.50}

\subsection*{Example 6: Roulette Bet}

\$1 bet on a single number in American roulette (38 slots):

\begin{tabular}{lll}
\toprule
Outcome & Net Gain (\$) & Probability \\
\midrule
Win & 35 & \(\frac{1}{38}\) \\
Lose & -1 & \(\frac{37}{38}\) \\
\bottomrule
\end{tabular}

\[
\begin{aligned}
E[X] &= 35 \cdot \frac{1}{38} + (-1) \cdot \frac{37}{38} \\
&= \frac{35 - 37}{38} = \frac{-2}{38} \approx -0.0526
\end{aligned}
\]

\textbf{Expected loss: 5.26 cents per \$1 bet}

\end{document}
