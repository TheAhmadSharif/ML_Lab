\documentclass{article}
\usepackage{amsmath}
\usepackage{amssymb}
\usepackage{geometry}
\geometry{margin=1in}

\title{Bayesian Analysis I: Fall 2024 Take-Home Assignment}
\author{}
\date{}

\begin{document}

\maketitle

\noindent\textbf{Instructions:}  
Be sure to include your codes, output, and detailed comments for each problem in your write-up. When submitting, convert your write-up into a single file (preferably in PDF) that includes your name and student number. You may use any books, references, and notes, but are not allowed to discuss these problems with any person other than the instructor until the due date. No credit will be given if any collaboration is detected.

\section*{Problem 1}
We aim to assess the efficacy of a COVID-19 vaccine based on data from the Pfizer Phase III clinical trial. The trial involves two groups: a placebo group and a vaccine group. The number of infected participants in each group is assumed to follow a Binomial distribution.

\begin{center}
\begin{tabular}{|c|c|c|c|}
\hline
\textbf{Group} & \textbf{Infected Participants} & \textbf{Severe Cases} & \textbf{Total Participants} \\
\hline
Placebo (Pfizer Study) & 162 & 9 & 18,325 \\
Vaccine (Pfizer Study) & 8 & 1 & 18,198 \\
\hline
\end{tabular}
\end{center}

Let $\theta_p$ represent the infection probability for the placebo group, and $\theta_v$ for the vaccine group, i.e., the probability that a randomly selected individual in the placebo group contracts the virus is $\theta_p$, while the probability that a randomly selected individual in the vaccine group contracts the virus is $\theta_v$. The efficacy of the vaccine is often defined as:
\[
E = 1 - \frac{\theta_v}{\theta_p}.
\]

Using these data, answer the following questions:
\begin{enumerate}
    \item[(a)] With a noninformative conjugate prior distribution for $\theta_p$, obtain the posterior distribution of $\theta_p$. Similarly, derive the posterior distribution of $\theta_v$.
    \item[(b)] Using all patients, conduct a Bayesian analysis of the efficacy of the vaccine, $E$. Specifically, determine whether the vaccine has efficacy at least 0.70 and quantify the uncertainty in the vaccine's efficacy using credible intervals.
    \item[(c)] Are the results from (b) sensitive to different priors (i.e., do they change a lot when you change the priors)? Analyze and compare the outcomes using 3-4 distinct priors of your choice. Summarize the results with a clearly labeled plot and table, and provide comments on your findings.
\end{enumerate}

\section*{Problem 2}
A study was conducted on 32 cars to explore the relationship between gasoline consumption and the weight of the car and engine sizes in cylinders.

For each car, we have observations on:
\begin{itemize}
    \item \textbf{mpg}: miles per gallon,
    \item \textbf{weight}: weight of the car,
    \item \textbf{sixcyl}: dummy variable for six cylinders (1 if six cylinders, 0 otherwise),
    \item \textbf{eightcyl}: dummy variable for eight cylinders (1 if eight cylinders, 0 otherwise).
\end{itemize}

\noindent The data is as follows:  
\textbf{mpg:} 21.0, 21.0, 22.8, 21.4, 18.7, 18.1, 14.3, 24.4, 22.8, 19.2, 17.8, 16.4, 17.3, 15.2, 10.4, 10.4, 14.7, 32.4, 30.4, 33.9, 21.5, 15.5, 15.2, 13.3, 19.2, 27.3, 26.0, 30.4, 15.8, 19.7, 15.0, 21.4.

\textbf{weight:} 2.620, 2.875, 2.320, 3.215, 3.440, 3.460, 3.570, 3.190, 3.150, 3.440, 3.440, 4.070, 3.730, 3.780, 5.250, 5.424, 5.345, 2.200, 1.615, 1.835, 2.465, 3.520, 3.435, 3.840, 3.845, 1.935, 2.140, 1.513, 3.170, 2.770, 3.570, 2.780.

\textbf{sixcyl:} 1, 1, 0, 1, 0, 1, 0, 0, 0, 1, 1, 0, 0, 0, 0, 0, 0, 0, 0, 0, 0, 0, 0, 0, 0, 0, 0, 0, 0, 1, 0, 0.

\textbf{eightcyl:} 0, 0, 0, 0, 1, 0, 1, 0, 0, 0, 0, 1, 1, 1, 1, 1, 1, 0, 0, 0, 0, 1, 1, 1, 1, 0, 0, 0, 1, 0, 1, 0.

We want to sample from the joint posterior distribution in the Normal linear regression:
\[
\text{mpg} = \beta_0 + \beta_1 \cdot \text{weight} + \beta_2 \cdot \text{sixcyl} + \beta_3 \cdot \text{eightcyl} + \varepsilon, \quad \varepsilon \sim N(0, \tau),
\]
with conjugate priors:
\[
\beta_i \sim N(0, 10000), \quad i = 0, \dots, 3, \quad \text{and} \quad 1/\tau \sim \text{Gamma}(0.01, 0.01).
\]

Answer the following questions:
\begin{enumerate}
    \item[(a)] Give the plots of the marginal posterior distributions for the parameters $\beta_1$, $\beta_2$, and $\beta_3$.
    \item[(b)] Construct 95\% equal tail probability intervals for each parameter and interpret them.
    \item[(c)] Investigate if the effect on mpg is different in cars with six cylinders compared to cars with eight cylinders.
    \item[(d)] Obtain the predictive distribution for a new 4-cylinder car with weight = 3.5. Provide comments.
\end{enumerate}

\noindent\textbf{Extra-Credit:}
\begin{enumerate}
    \item[(e)] Add weight as a random effect: $u_i \sim N(0, \sigma_u^2)$. How do posterior results change compared to the fixed-effect model?
\end{enumerate}

\end{document}
