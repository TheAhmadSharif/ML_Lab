\documentclass{article}
\usepackage{amsmath}
\usepackage{amsfonts}
\usepackage{geometry}
\geometry{a4paper, margin=1in}

\begin{document}

\section*{Manual Explanation of Cluster Evaluation Methods with a Native Data Example}

\subsection*{1. Elbow Method}

\subsubsection*{Native Data Example}

Consider the following 1D data points: $\{1, 2, 4, 5, 10, 11\}$. We want to find the optimal number of clusters $k$.


WCSS: within cluster sum of squares

\begin{enumerate}
    \item \textbf{k = 1:}
    \begin{itemize}
        \item Centroid: $\mu_1 = (1+2+4+5+10+11)/6 = 33/6 = 5.5$
        \item WCSS: $(1-5.5)^2 + (2-5.5)^2 + (4-5.5)^2 + (5-5.5)^2 + (10-5.5)^2 + (11-5.5)^2$
        $= (-4.5)^2 + (-3.5)^2 + (-1.5)^2 + (-0.5)^2 + (4.5)^2 + (5.5)^2$
        $= 20.25 + 12.25 + 2.25 + 0.25 + 20.25 + 30.25 = 85.5$
    \end{itemize}
    \item \textbf{k = 2:} Let's assume clusters $C_1 = \{1, 2, 4\}$ and $C_2 = \{5, 10, 11\}$ (a possible split).
    \begin{itemize}
        \item Centroid $C_1$: $\mu_1 = (1+2+4)/3 = 7/3 \approx 2.33$
        \item Centroid $C_2$: $\mu_2 = (5+10+11)/3 = 26/3 \approx 8.67$
        \item WCSS: $((1-2.33)^2 + (2-2.33)^2 + (4-2.33)^2) + ((5-8.67)^2 + (10-8.67)^2 + (11-8.67)^2)$
        $\approx (1.77 + 0.11 + 2.79) + (13.47 + 1.77 + 5.34) = 4.67 + 20.58 = 25.25$
    \end{itemize}
    \item \textbf{k = 3:} Let's assume clusters $C_1 = \{1, 2\}$, $C_2 = \{4, 5\}$, $C_3 = \{10, 11\}$.
    \begin{itemize}
        \item Centroid $C_1$: $\mu_1 = (1+2)/2 = 1.5$
        \item Centroid $C_2$: $\mu_2 = (4+5)/2 = 4.5$
        \item Centroid $C_3$: $\mu_3 = (10+11)/2 = 10.5$
        \item WCSS: $((1-1.5)^2 + (2-1.5)^2) + ((4-4.5)^2 + (5-4.5)^2) + ((10-10.5)^2 + (11-10.5)^2)$
        $= (0.25 + 0.25) + (0.25 + 0.25) + (0.25 + 0.25) = 0.5 + 0.5 + 0.5 = 1.5$
    \end{itemize}
\end{enumerate}

If we were to plot WCSS against $k$ (1: 85.5, 2: 25.25, 3: 1.5), we would observe a significant drop from $k=1$ to $k=2$, and then a smaller drop from $k=2$ to $k=3$. The "elbow" is likely at $k=2$, suggesting two clusters might be optimal for this data.

\subsection*{2. Silhouette Score}

\subsubsection*{Native Data Example}

Using the same 1D data $\{1, 2, 4, 5, 10, 11\}$ and the $k=2$ clustering $C_1 = \{1, 2, 4\}$ and $C_2 = \{5, 10, 11\}$.

\begin{enumerate}
    \item \textbf{Point 1 (Cluster 1):}
    \begin{itemize}
        \item $a(1) = (|1-2| + |1-4|) / 2 = (1 + 3) / 2 = 2$
        \item $b(1)$: Average distance to Cluster 2: $(|1-5| + |1-10| + |1-11|) / 3 = (4 + 9 + 10) / 3 = 23/3 \approx 7.67$
        \item $s(1) = (7.67 - 2) / \max(2, 7.67) = 5.67 / 7.67 \approx 0.74$
    \end{itemize}
    \item \textbf{Point 4 (Cluster 1):}
    \begin{itemize}
        \item $a(4) = (|4-1| + |4-2|) / 2 = (3 + 2) / 2 = 2.5$
        \item $b(4)$: Average distance to Cluster 2: $(|4-5| + |4-10| + |4-11|) / 3 = (1 + 6 + 7) / 3 = 14/3 \approx 4.67$
        \item $s(4) = (4.67 - 2.5) / \max(2.5, 4.67) = 2.17 / 4.67 \approx 0.46$
    \end{itemize}
    \item \textbf{Point 5 (Cluster 2):}
    \begin{itemize}
        \item $a(5) = (|5-10| + |5-11|) / 2 = (5 + 6) / 2 = 5.5$
        \item $b(5)$: Average distance to Cluster 1: $(|5-1| + |5-2| + |5-4|) / 3 = (4 + 3 + 1) / 3 = 8/3 \approx 2.67$
        \item $s(5) = (2.67 - 5.5) / \max(5.5, 2.67) = -2.83 / 5.5 \approx -0.51$
    \end{itemize}
    \item \textbf{Point 11 (Cluster 2):}
    \begin{itemize}
        \item $a(11) = (|11-5| + |11-10|) / 2 = (6 + 1) / 2 = 3.5$
        \item $b(11)$: Average distance to Cluster 1: $(|11-1| + |11-2| + |11-4|) / 3 = (10 + 9 + 7) / 3 = 26/3 \approx 8.67$
        \item $s(11) = (8.67 - 3.5) / \max(3.5, 8.67) = 5.17 / 8.67 \approx 0.60$
    \end{itemize}
\end{enumerate}

To get the overall Silhouette Score for $k=2$ with this split, we would calculate $s(2)$ and $s(10)$ similarly and then average all six silhouette coefficients. A higher average score for a particular $k$ would suggest a better clustering. You would repeat this process for different values of $k$ and choose the one with the highest average Silhouette Score.

\end{document}
