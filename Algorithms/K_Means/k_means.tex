\documentclass{article}
\usepackage{amsmath}
\usepackage{geometry}
\usepackage{booktabs}
\usepackage{graphicx}

\geometry{margin=1in}

\title{Manual K-Means Clustering: Step-by-Step Calculation}
\author{}
\date{}

\begin{document}

\maketitle

\section*{Data Points}

We are given the following points:

\begin{center}
\begin{tabular}{cc}
\toprule
Point & Coordinates \\
\midrule
A & (1, 1) \\
B & (1, 4) \\
C & (5, 1) \\
D & (5, 4) \\
\bottomrule
\end{tabular}
\end{center}

We will explore two different centroid initializations and their effects on clustering results.

\section*{Initialization 1}

\textbf{Initial Centroids:}
\begin{itemize}
    \item Centroid 1 (C1): Point A $\rightarrow (1, 1)$
    \item Centroid 2 (C2): Point D $\rightarrow (5, 4)$
\end{itemize}

\subsection*{Step 1: Distance Calculation}

The Euclidean distance formula is:
\[
d((x_1, y_1), (x_2, y_2)) = \sqrt{(x_1 - x_2)^2 + (y_1 - y_2)^2}
\]

\textbf{Distances to C1 = (1, 1):}
\begin{align*}
d(A, C1) &= \sqrt{(1 - 1)^2 + (1 - 1)^2} = 0 \\
d(B, C1) &= \sqrt{(1 - 1)^2 + (4 - 1)^2} = \sqrt{9} = 3 \\
d(C, C1) &= \sqrt{(5 - 1)^2 + (1 - 1)^2} = \sqrt{16} = 4 \\
d(D, C1) &= \sqrt{(5 - 1)^2 + (4 - 1)^2} = \sqrt{25} = 5
\end{align*}

\textbf{Distances to C2 = (5, 4):}
\begin{align*}
d(A, C2) &= \sqrt{(1 - 5)^2 + (1 - 4)^2} = \sqrt{25} = 5 \\
d(B, C2) &= \sqrt{(1 - 5)^2 + (4 - 4)^2} = \sqrt{16} = 4 \\
d(C, C2) &= \sqrt{(5 - 5)^2 + (1 - 4)^2} = \sqrt{9} = 3 \\
d(D, C2) &= \sqrt{(5 - 5)^2 + (4 - 4)^2} = 0
\end{align*}

\subsection*{Step 2: Assign to Nearest Centroid}

\begin{center}
\begin{tabular}{cccc}
\toprule
Point & Distance to C1 & Distance to C2 & Assigned Cluster \\
\midrule
A & 0 & 5 & C1 \\
B & 3 & 4 & C1 \\
C & 4 & 3 & C2 \\
D & 5 & 0 & C2 \\
\bottomrule
\end{tabular}
\end{center}

\subsection*{Step 3: Recalculate Centroids}

\textbf{Cluster 1:} A(1,1), B(1,4)
\[
C1 = \left( \frac{1 + 1}{2}, \frac{1 + 4}{2} \right) = (1, 2.5)
\]

\textbf{Cluster 2:} C(5,1), D(5,4)
\[
C2 = \left( \frac{5 + 5}{2}, \frac{1 + 4}{2} \right) = (5, 2.5)
\]

\textbf{Final Clusters:}
\[
\begin{cases}
\text{C1} \rightarrow \{ A, B \} \\
\text{C2} \rightarrow \{ C, D \}
\end{cases}
\]

\section*{Initialization 2}

\textbf{Initial Centroids:}
\begin{itemize}
    \item Centroid 1 (C1): Point C $\rightarrow (5, 1)$
    \item Centroid 2 (C2): Point D $\rightarrow (5, 4)$
\end{itemize}

\subsection*{Step 1: Distance Calculation}

\textbf{Distances to C1 = (5, 1):}
\begin{align*}
d(A, C1) &= \sqrt{(1 - 5)^2 + (1 - 1)^2} = \sqrt{16} = 4 \\
d(B, C1) &= \sqrt{(1 - 5)^2 + (4 - 1)^2} = \sqrt{25} = 5 \\
d(C, C1) &= 0 \\
d(D, C1) &= \sqrt{(5 - 5)^2 + (4 - 1)^2} = \sqrt{9} = 3
\end{align*}

\textbf{Distances to C2 = (5, 4):}
\begin{align*}
d(A, C2) &= \sqrt{(1 - 5)^2 + (1 - 4)^2} = \sqrt{25} = 5 \\
d(B, C2) &= \sqrt{(1 - 5)^2 + (4 - 4)^2} = \sqrt{16} = 4 \\
d(C, C2) &= \sqrt{(5 - 5)^2 + (1 - 4)^2} = \sqrt{9} = 3 \\
d(D, C2) &= 0
\end{align*}

\subsection*{Step 2: Assign to Nearest Centroid}

\begin{center}
\begin{tabular}{cccc}
\toprule
Point & Distance to C1 & Distance to C2 & Assigned Cluster \\
\midrule
A & 4 & 5 & C1 \\
B & 5 & 4 & C2 \\
C & 0 & 3 & C1 \\
D & 3 & 0 & C2 \\
\bottomrule
\end{tabular}
\end{center}

\subsection*{Step 3: Recalculate Centroids}

\textbf{Cluster 1:} A(1,1), C(5,1)
\[
C1 = \left( \frac{1 + 5}{2}, \frac{1 + 1}{2} \right) = (3, 1)
\]

\textbf{Cluster 2:} B(1,4), D(5,4)
\[
C2 = \left( \frac{1 + 5}{2}, \frac{4 + 4}{2} \right) = (3, 4)
\]

\textbf{Final Clusters:}
\[
\begin{cases}
\text{C1} \rightarrow \{ A, C \} \\
\text{C2} \rightarrow \{ B, D \}
\end{cases}
\]

\section*{Summary}

\begin{center}
\begin{tabular}{ccc}
\toprule
Initialization & Cluster 1 & Cluster 2 \\
\midrule
(A, D) & A, B & C, D \\
(C, D) & A, C & B, D \\
\bottomrule
\end{tabular}
\end{center}

\noindent This demonstrates that the outcome of K-Means clustering can vary significantly depending on the initial centroid selection.


\section{K-Mean++}

\section*{Dataset}

We are given the following points:

\begin{center}
\begin{tabular}{ll}
\toprule
Point & Coordinates \\
\midrule
A & (1, 1) \\
B & (1, 4) \\
C & (5, 1) \\
D & (5, 4) \\
\bottomrule
\end{tabular}
\end{center}

We want to cluster the data into \( k = 2 \) clusters using the K-Means++ algorithm.

\section*{Step 1: Randomly Choose the First Centroid}

Assume we randomly select point B \( (1, 4) \) as the first centroid:
\[
\text{Centroid } C_1 = (1, 4)
\]

\section*{Step 2: Compute Squared Distances \( D(x)^2 \) to the Nearest Centroid}

\[
\begin{aligned}
D(A)^2 &= (1 - 1)^2 + (1 - 4)^2 = 0 + 9 = 9 \\
D(C)^2 &= (5 - 1)^2 + (1 - 4)^2 = 16 + 9 = 25 \\
D(D)^2 &= (5 - 1)^2 + (4 - 4)^2 = 16 + 0 = 16 \\
\end{aligned}
\]

\begin{center}
\begin{tabular}{lll}
\toprule
Point & \( D(x)^2 \) & Distance \\
\midrule
A & 9 & 3.0 \\
C & 25 & 5.0 \\
D & 16 & 4.0 \\
\bottomrule
\end{tabular}
\end{center}

\section*{Step 3: Select the Next Centroid Probabilistically}

\[
\text{Total} = 9 + 25 + 16 = 50
\]
\[
P(A) = \frac{9}{50} = 0.18, \quad P(C) = \frac{25}{50} = 0.50, \quad P(D) = \frac{16}{50} = 0.32
\]

Suppose the random selection gives point C \( (5, 1) \) as the next centroid:

\[
\text{Centroid } C_2 = (5, 1)
\]

\section*{Step 4: Assign Each Point to the Nearest Centroid}

\[
\begin{aligned}
D(A, C_1)^2 &= (1 - 1)^2 + (1 - 4)^2 = 9 \\
D(A, C_2)^2 &= (1 - 5)^2 + (1 - 1)^2 = 16 \Rightarrow \text{Assign to } C_1 \\
\\
D(B, C_1)^2 &= 0, \quad D(B, C_2)^2 = (1 - 5)^2 + (4 - 1)^2 = 25 \Rightarrow \text{Assign to } C_1 \\
\\
D(C, C_1)^2 &= 25, \quad D(C, C_2)^2 = 0 \Rightarrow \text{Assign to } C_2 \\
\\
D(D, C_1)^2 &= 16, \quad D(D, C_2)^2 = 9 \Rightarrow \text{Assign to } C_2 \\
\end{aligned}
\]

\begin{center}
\begin{tabular}{llll}
\toprule
Point & Distance to \( C_1 \) & Distance to \( C_2 \) & Cluster \\
\midrule
A & 3.0 & 4.0 & \( C_1 \) \\
B & 0.0 & 5.0 & \( C_1 \) \\
C & 5.0 & 0.0 & \( C_2 \) \\
D & 4.0 & 3.0 & \( C_2 \) \\
\bottomrule
\end{tabular}
\end{center}

\section*{Step 5: Recalculate Centroids}

\subsection*{Cluster 1 (A, B)}
\[
x = \frac{1 + 1}{2} = 1, \quad y = \frac{1 + 4}{2} = 2.5 \Rightarrow \text{New } C_1 = (1, 2.5)
\]

\subsection*{Cluster 2 (C, D)}
\[
x = \frac{5 + 5}{2} = 5, \quad y = \frac{1 + 4}{2} = 2.5 \Rightarrow \text{New } C_2 = (5, 2.5)
\]

\section*{Final Clusters After 1 Iteration}

\begin{itemize}
    \item Cluster 1: A (1,1), B (1,4)
    \item Cluster 2: C (5,1), D (5,4)
\end{itemize}

Updated centroids:
\[
C_1 = (1, 2.5), \quad C_2 = (5, 2.5)
\]

\end{document}
